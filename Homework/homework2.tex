\documentclass{article}

%
% 引入模板的style文件
%
\usepackage{homework}
\usepackage{xeCJK} %中文包,正常显示中文
\usepackage{ctex} %支持中文,不加这个包,参考文献几个大字显示为英文且出警告
\usepackage{url}
%
% 封面
%

\title{
	\includegraphics[scale = 0.45]{images/title/ucas-logo1.png}\\
    \vspace{1in}
    \textmd{\textbf{\hmwkClass\ \hmwkTitle}}\\
    \textmd{\textbf{\hmwkSubTitle}}\\
    \normalsize\vspace{0.1in}\small{\hmwkCompleteTime }\\
    \vspace{0.1in}\large{\textit{\hmwkClassInstructor\ }}\\
    \vspace{3in}
}

\author{\hmwkAuthorName \\ 
	\hmwkAuthorStuID}
\date{}

\renewcommand{\part}[1]{\textbf{\large Part \Alph{partCounter}}\stepcounter{partCounter}\\}


%
% 正文部分
%
\begin{document}


\maketitle


%\include{chapters/ch01}


\pagebreak

\begin{homeworkProblem}
\textbf{1.现有一组数据,实际的恶意用户和非恶意用户数量分别为1109和14891。通过在线恶意用户检测模型对这些数据进行恶意用户检测,预测结果为:恶意用户和非恶意用户数量分别为2113和13887,其中准确预测的恶意用户数量为891,准确预测的非恶意用户数量为13669。(1).对混淆矩阵表~\ref{tab:F}中的空缺值进行填充;(2).请依据混淆矩阵分别计算模型的评价指标Precision,Recall,F-measure。}

{\color{blue}\textbf{答:(1)混淆矩阵如下:}\\

\begin{table}[htbp!]\centering
	\caption{\textbf{混淆矩阵}}
	\label{tab:F}
\begin{tabular}{|c|ccc|c|}
\hline
& \multicolumn{3}{c|}{\textbf{预测类别}}    & {\textbf{合计}}    \\ \hline
\multirow{4}{*}{\textbf{\begin{tabular}[c]{@{}l@{}}实\\ 际 \\ 类\\ 别\end{tabular}}} & \multicolumn{1}{l|}{}  & \multicolumn{1}{l|}{\textbf{类别=恶意用户}} & \textbf{类别=非恶意用户} &     \\ \cline{2-5} 
& \multicolumn{1}{c|}{\textbf{类别=恶意用户}}  & \multicolumn{1}{c|}{\color{red}TP=}  & {\color{red}FN=}  & 1109  \\ \cline{2-5} 
& \multicolumn{1}{c|}{\textbf{类别=非恶意用户}} & \multicolumn{1}{c|}{\color{red}FP=}  & {\color{red}TN=}         & 14891\\ \cline{2-5} 
& \multicolumn{1}{c|}{\textbf{合计}}  & \multicolumn{1}{c|}{2113}    &{13887}  &{16000}  \\ \hline
\end{tabular}
\end{table}

(2)\textbf{Precision,Recall,F-measure的计算过程如下:}\\
\begin{equation}\label{equ:sigmod}
\begin{aligned}
Precision=\frac{}{},\\
Recall=\frac{}{},\\
F-measure=\frac{}{}.
\end{aligned}
\end{equation}
}
\end{homeworkProblem}


\begin{homeworkProblem}
	\textbf{2. 请解释岭回归和Lasso回归的原理和应用场景。}\\
{\color{blue}\textbf{答:}
\begin{itemize}
\item{\textbf{岭回归}:}
\item{\textbf{Lasso回归}:}
\end{itemize}

}	
\end{homeworkProblem}


\begin{homeworkProblem}
	\textbf{3. 假设你有一组房屋的特征数据(如面积、卧室数量等)和对应的售价数据,你想使用机器学习回归模型来预测房屋的售价。请描述你会如何处理这个问题,并选择合适的回归算法。}\\
{\color{blue}\textbf{答:}


}	
\end{homeworkProblem}




% 引用文献 根据要求引用
%\bibliographystyle{unsrt}  % unsrt:根据引用顺序编号
%\bibliography{refs}


\end{document}
